\documentclass[a4paper,10pt]{article} % type, taille police

\usepackage[utf8]{inputenc} % encodage
\usepackage[T1]{fontenc} % encodage
%\usepackage[french]{babel} % gestion du français
\usepackage{amssymb} % symboles mathématiques
\usepackage{textcomp} % flèche,  intervalle
\usepackage{stmaryrd} % intervalle entiers

\usepackage[left=3cm,right=3cm,top=3cm,bottom=3cm]{geometry} % marges
\usepackage[hidelinks]{hyperref} % sommaire interactif dans un pdf
\usepackage[nottoc, notlof, notlot]{tocbibind} % affichage des références dans la table des matières (?)
\usepackage{float} % placement des figures
\usepackage[toc,page]{appendix} % ajout d'annexes
\usepackage{amsthm} % format des déf, prop...
\usepackage{amsmath} % matrices, ...
\usepackage{multirow} % fusionner cellules verticalement

\usepackage[french,ruled,vlined,linesnumbered]{algorithm2e} % affichage d'algorithmes
\usepackage{tikz} % affichage de schémas
\usepackage{graphicx} % affichage d'images
\usepackage{url} % inclure des urls
\usepackage{bbold} % fonction caractéristique 1

\renewcommand{\appendixtocname}{Annexes} % renommage annexes
\renewcommand{\appendixpagename}{Annexes}

%\Pbrovidecommand{\SetAlgoLined}{\SetLine} % paramètre pour algorithm2e
%\Pbrovidecommand{\DontPrintSemicolon}{\dontprintsemicolon}  % paramètre pour algorithm2e

\definecolor{bgreen}{rgb}{0.30,0.70,0}

\theoremstyle{definition} % pas d'italique pour le format des déf, prop...
\newtheorem{thm}{Theorem} % \begin{thm} \end{thm}
\newtheorem{prop}[thm]{Property}
\newtheorem{cor}[thm]{Corollary}
\newtheorem{defi}[thm]{Definition}
\newtheorem{ex}[thm]{Example}
\newtheorem{lem}[thm]{Lemma}
\newtheorem{rmk}[thm]{Remark}
\newtheorem{conj}[thm]{Conjecture}

\newcommand{\ddelta}{\delta^\dag}

%#############################################################################################################%
%#############################################################################################################%
%#############################################################################################################%

\title{Distributing the Heat Equation}
\author{Tom Cornebize \and Yassine Hamoudi}
\date{Sunday, December 07}

\begin{document}

\maketitle

%#############################################################################################################%
%#############################################################################################################%
%#############################################################################################################%

\section{Question 1}

\begin{lem}
  \label{nextStep}
  $N^2$ applications of function $\delta$ are necessary to compute $X^t$ from $X^{t-1}$.
\end{lem}

\begin{proof}
 Each cell $X^{t}_{i,j}$ needs one application of $\delta$ to be computed from $X^{t-1}_{i,j}$. There are $N^2$ cells, so $N^2$ applications of $\delta$ are needed.
\end{proof}

\begin{prop}
  $tN^2$ applications of function $\delta$ are necessary to compute $X^t$ on \textlbrackdbl $0,N-1$ \textrbrackdbl$^2$.
\end{prop}

\begin{proof}
 $X^t$ is obtained after $t$ applications of $\ddelta$ on $X^0$. Each application needs $N^2$ calls to $\delta$ according to lemma \ref{nextStep}. The whole computation needs $tN^2$ applications of $\delta$.
\end{proof}

%#############################################################################################################%

\section{Question 2}

We associate one processor per cell ($N^2$ processors are needed). Each processor $p_{i,j}$ stores at time $t$ the value of cell $X^{t}_{i,j}$.

At time $t$, each processor sends its value to its 8 neighbours and receives their values in parallel. Then each processor updates $X^{t}_{i,j}$ to $X^{t+1}_{i,j}$. 

$\longrightarrow$ à changer :(, le nombre de processeurs est un paramètre donné en entrée ($< N^2$ à priori)

%#############################################################################################################%



%#############################################################################################################%
%#############################################################################################################%
%#############################################################################################################%

\end{document}
